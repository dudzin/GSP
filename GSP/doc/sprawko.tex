\documentclass[11pt,a4paper]{article}

\usepackage[compact]{titlesec}
\titleformat*{\section}{\Large\bfseries}
\titleformat*{\subsection}{\large\bfseries}
\titleformat*{\subsubsection}{\normalsize\bfseries}
\titlespacing{\section}{-15pt}{*5}{*3}
\titlespacing{\subsection}{0pt}{*3}{*1.5}
\titlespacing{\subsubsection}{8pt}{*2}{*1}
\setlength{\parskip}{0cm}
\setcounter{secnumdepth}{3}
\setcounter{tocdepth}{2}
	
% jezyk i czcionki polskie
\usepackage{polski}
\usepackage[utf8x]{inputenc}
% zmiana wygladu naglowkow sekcji
	%\usepackage{sectsty} 
	%\sectionfont{\large}
%ustawienie marginesĂłw
\usepackage{anysize}
\marginsize{3cm}{3cm}{0.5cm}{1cm}
%inne pakiety
\usepackage{graphicx} %grafiki
\usepackage{color}
\usepackage{amsmath} % matematyka
\usepackage{amssymb} % symbole, np. triangleeq
\usepackage{float}
\usepackage{indentfirst}

\title{ \textbf{Metody eksploracji danych - projekt} \\ \Large Dokumentacja}
\author{Przemysław Barcikowski, Dariusz Dudziński}

\begin{document}
\maketitle
\section{Zadanie}
\subsection{Temat}
\paragraph{} Wyszukiwanie uogolnionych wzorcow sekwencyjnych (Generalized Sequential Patterns)
\subsection{Cel projektu}
\paragraph{} Celem niniejszego projektu jest stworzenie aplikacji spełniającej poniższe wymagania:
\begin{enumerate}
\item Aplikacja służy do badania sekwencyjnych reguł asocjacyjnych na podstawie danych z giełdy,
\item Aplikacja operuje na csv, gdzie np pierwsza kolumna to nazwa serii, druga to data. Kolejne atrybuty będą wykorzystywane do samego wyliczania sekwencji. Powinny to być wartości liczbowe,
\item Aplikacja automatycznie buduje taksonomię na atrybutach liczbowych w postaci
\begin{itemize}
\item zaokrąglenie do pełnej wartości,
\item wartość dodatnia/ujemna.
\end{itemize}
\item Aplikacja operuje z zadanym przez użytkownika oknem czasowym (sliding window), ograniczeniami czasowymi (min i max różnica między wystąpieniami w serii) i minimalnym wsparciem
\item Aplikacja generuje sekwencyjne reguły asocjacyjne,
\item Napisana w języku Java,
\item Badanie będzie miało na celu odnalezienie reguł z danych.
\end{enumerate}

\section{Rozwiązanie}
\label{sec:rozwiazanie}
Do rozwiązania wyżej postawionego problemu został wykorzystany algorytm Generalized Sequential Patterns (GSP) \cite{bib:GSP}. Jako dane do testowania oraz eksperymentów wybrano notowania giełdowe indeksu Dow Jones, pobrane ze strony \cite{bib:DowJones}.

\section{Założenia programu}

\subsection{Funkcjonalności aplikacji}
\subsection{Konfiguracja parametrów programu}
\subsection{Wybrana technologia i wydajność}

\section{Implementacja}

\section{Testy}

\section{Eksperymenty i wyniki}
\nocite{*}
\bibliography{bibl}
\bibliographystyle{plain}

\end{document}
